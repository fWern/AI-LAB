\documentclass[sigconf, screen, authorversion]{acmart}
\usepackage{booktabs}
\usepackage{tabularx}
\usepackage{hhline}
\usepackage{graphicx}
\usepackage[compatibility]{scikgtex}
\usepackage{cleveref}
\usepackage{metalogo}
\usepackage{verbatim}
\usepackage{listings}
\usepackage{csquotes}
\usepackage{tablefootnote}
\usepackage{float}
\usepackage{dblfloatfix}

\makeatletter
\renewcommand\@formatdoi[1]{\ignorespaces}
\def\@copyrightspace{\relax}
\makeatother

\crefformat{footnote}{#2\footnotemark[#1]#3}

\lstset{breaklines=true, basicstyle=\small\ttfamily}

\setcopyright{none}
\begin{CCSXML}
	<ccs2012>
	<concept>
	<concept_id>10010405.10010497.10010510.10010512</concept_id>
	<concept_desc>Applied computing~Markup languages</concept_desc>
	<concept_significance>500</concept_significance>
	</concept>
	<concept>
	<concept_id>10010405.10010497.10010500.10010503</concept_id>
	<concept_desc>Applied computing~Document metadata</concept_desc>
	<concept_significance>500</concept_significance>
	</concept>
	<concept>
	<concept_id>10010405.10010497.10010510.10010513</concept_id>
	<concept_desc>Applied computing~Annotation</concept_desc>
	<concept_significance>500</concept_significance>
	</concept>
	</ccs2012>
\end{CCSXML}

\ccsdesc[500]{Applied computing~Markup languages}
\ccsdesc[500]{Applied computing~Document metadata}
\ccsdesc[500]{Applied computing~Annotation}

%\acmDOI{xx.xxx/xxx_x}

% ISBN
%\acmISBN{978-1-4503-8713-2/22/04}

%Conference
\acmConference[JCDL 2023]{ACM/IEEE Joint Conference On Digital Libraries}{June 26 - 30, 2023}{Santa Fe, New Mexico, USA}
\acmYear{2023}
\copyrightyear{2023}

%\acmArticle{4}
%\acmPrice{15.00}

\begin{document}
	%
	\metatitle{\title{SciKGTeX - A \LaTeX{} Package to Semantically Annotate Contributions in Scientific Publications}}
	\renewcommand{\shorttitle}{SciKGTeX}
	\author{Christof Bless}
	\orcid{0000-0001-9778-8495}
	\affiliation{%
		\institution{Lucerne University of Applied Sciences}
		\city{Lucerne}
		\country{Switzerland}
	}
	\email{christofbless@gmail.com}
	
	\author{Ildar Baimuratov}
	\orcid{0000-0002-6573-131X}
	\affiliation{%
		\institution{L3S Research Center\\ Leibniz University Hannover}
		\city{Hannover}
		\country{Germany}
	}
	\email{baimuratov.i@gmail.com}
	
	\author{Oliver Karras}
	\orcid{0000-0001-5336-6899}
	\affiliation{%
		\institution{TIB - Leibniz Information Centre for Science and Technology}
		\city{Hannover}
		\country{Germany}
	}
	\email{oliver.karras@tib.eu}
	
	\metaauthor*{\uri{https://orcid.org/0000-0001-9778-8495}{Christof Bless}}
	\metaauthor*{\uri{https://orcid.org/0000-0002-6573-131X}{Ildar Baimuratov}}
	\metaauthor*{\uri{https://orcid.org/0000-0001-5336-6899}{Oliver Karras}}
	\researchfield*{\uri{https://orkg.org/resource/R278}{Information Science}}
	
	\begin{abstract}
		Scientific knowledge graphs have been proposed as a solution to structure the content of research publications in a machine-actionable way and enable more efficient, computer-assisted workflows for many research activities.
		\researchproblem*{crowd-sourcing for scientific knowledge graphs}
		Crowd-sourcing approaches are used frequently to build and maintain such scientific knowledge graphs.
		To contribute to scientific knowledge graphs, researchers need simple and easy-to-use solutions to generate new knowledge graph elements and establish the practice of semantic representations in scientific communication.
		In this paper, we present a \objective{workflow for authors of scientific documents to specify their contributions} with a \LaTeX{} package, called SciKGTeX, and upload them to a scientific knowledge graph.
		\objective*{automatically upload contributions to a knowledge graph}
		\method*{latex}
		\method*{luatex}
		The SciKGTeX package allows authors of scientific publications to mark the main contributions of their work directly in \LaTeX{} source files. The package embeds marked contributions as metadata into the generated PDF document, from where they can be extracted automatically and imported into a scientific knowledge graph, such as the ORKG.
		This workflow is simpler and faster than current approaches, which make use of external web interfaces for data entry.
		Our user evaluation shows that SciKGTeX is easy to use, with a \result{score of 79 out of 100 on the System Usability Scale}, as participants of the study needed only \result{7 minutes on average to annotate the main contributions} on a sample abstract of a published paper.
		Further testing shows that the embedded contributions can be successfully uploaded to ORKG within ten seconds.
		\conclusion{SciKGTeX simplifies the process of manual semantic annotation of research contributions in scientific articles. Our workflow demonstrates how a scientific knowledge graph can automatically ingest research contributions from document metadata.}
	\end{abstract}
	\keywords{Semantic Annotation, \LaTeX{}, FAIR data, Scientific Knowledge Graphs.}
	
	\maketitle              
	
	\input{sections/introduction}
	\input{sections/related}
	
	\input{sections/approach}
	\input{sections/implementation}
	\input{sections/evaluation}
	
	\input{sections/discussion}
	\input{sections/conclusion}
	
	\section*{Acknowledgement}
	The authors would like to thank the Federal Government and the Heads of Government of the Länder, as well as the Joint Science Conference (GWK), for their funding and support within the framework of the NFDI4Ing consortium. This work was partially funded by the German Research Foundation (DFG) - project number 442146713, by the European Research Council for the project ScienceGRAPH (Grant agreement ID: 819536), and by the TIB – Leibniz Information Centre for Science and Technology. 
	
	
	\bibliographystyle{ACM-Reference-Format}
	%\bibliographystyle{bib_style}
	\bibliography{biblio}
	
	
\end{document}
