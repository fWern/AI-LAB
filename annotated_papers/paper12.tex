\documentclass{article}

\usepackage[colorlinks=true, allcolors=blue]{hyperref}
\usepackage[compatibility]{scikgtex}

\title{
% add title to metadata
  \metatitle{Enhancing Candidate Link Generation for Requirements Tracing:
The Cluster Hypothesis Revisited}
}
\author{
% add author to metadata
  \metaauthor{Nan Niu}
  \and
  \metaauthor{Anas Mahmoud}
}

\begin{document}

\maketitle

\begin{abstract}
\researchproblem{Modern requirements tracing tools employ \uri{https://orkg.org/resource/R591027}{information retrieval} methods to automatically generate candidate links. Due to the inherent trade-off between \uri{https://orkg.org/resource/R6557}{recall} and \uri{https://orkg.org/resource/R6009}{precision}, such methods cannot achieve a high coverage without also retrieving a great number of false positives, causing a significant drop in result accuracy}. In this paper, we propose an \objective{approach to improving the quality of candidate link generation for the requirements tracing process}. We base our research on the \method{cluster hypothesis which suggests that correct and incorrect links can be grouped in high-quality and low-quality clusters respectively}. Result accuracy can thus be enhanced by identifying and filtering out low-quality clusters. We describe our approach by investigating three open-source datasets, and further evaluate our work through an industrial study. The results show that our \result{approach outperforms a baseline pruning strategy and that improvements are still possible}.
\end{abstract}

\end{document}
