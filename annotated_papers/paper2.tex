\documentclass{article}

\usepackage[colorlinks=true, allcolors=blue]{hyperref}
\usepackage[compatibility]{scikgtex}

\title{
% add title to metadata
  \metatitle{Application of Machine Learning in Predicting
Performance for Computer Engineering Students:
A Case Study}
}
\author{
% add author to metadata
  \metaauthor{Md Arju Hossain}
  \and
  \metaauthor{Md Habibur Rahman}
  \and
  \metaauthor{Habiba Sultana}
  \and
  \metaauthor{Asif Ahsan}
  \and
  \metaauthor{Saiful Islam Rayhan}
  \and
  \metaauthor{Md Imran Hasan}
  \and
  \metaauthor{Md Sohel}
  \and
  \metaauthor{Pratul Dipta Somadder}
  \and
  \metaauthor{Mahammad Ali Moni}
}
\researchfield*{\uri{https://orkg.org/resource/R104}{Bioinformatics}}

\begin{document}

\maketitle

\begin{abstract}
\researchproblem{With high inflammatory states from both \uri{https://orkg.org/resource/R212767}{COVID-19} and HIV conditions further result in complications. The ongoing confrontation between these two viral infections can be avoided by adopting suitable management measures}. The aim of this study was to \objective{figure out the pharmacological mechanism behind apigenin’s role in the synergetic effects of COVID-19 to the progression of HIV patients}. We employed \method{computer-aided methods to uncover similar biological targets and signaling pathways associated with COVID-19 and HIV, along with bioinformatics and network pharmacology techniques to assess the synergetic effects of apigenin on COVID-19 to the progression of HIV, as well as pharmacokinetics analysis to examine apigenin’s safety in the human body}. \result{Stress-responsive, membrane receptor, and induction pathways were mostly involved in gene ontology (GO) pathways, whereas apoptosis and inflammatory pathways were significantly associated in the Kyoto encyclopedia of genes and genomes (KEGG)}. The \result{top 20 hub genes were detected utilizing the shortest path ranked by degree method and protein-protein interaction (PPI), as well as molecular docking and molecular dynamics simulation were performed, revealing apigenin’s strong interaction with hub proteins (MAPK3, RELA, MAPK1, EP300, and AKT1)}. \result{Moreover, the pharmacokinetic features of apigenin revealed that it is an effective therapeutic agent with minimal adverse effects, for instance, hepatoxicity. Synergetic effects of COVID-19 on the progression of HIV may still be a danger to global public health}. Consequently, \conclusion{advanced solutions are required to give valid information regarding apigenin as a suitable therapeutic agent for the management of COVID-19 and HIV synergetic effects}. However, the findings have yet to be confirmed in patients, suggesting more in vitro and in vivo studies.
\end{abstract}

\section{Introduction}

\end{document}
