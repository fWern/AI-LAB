\documentclass{article}

\usepackage[colorlinks=true, allcolors=blue]{hyperref}
\usepackage[compatibility]{scikgtex}

\title{
% add title to metadata
  \metatitle{A Little Bird Told Me: Mining Tweets for
Requirements and Software Evolution}
}
\author{
% add author to metadata
  \metaauthor{Emitza Guzman}
  \and
  \metaauthor{Mohamed Ibrahim}
  \and
  \metaauthor{Martin Glinz}
}
\researchfield*{\uri{https://orkg.org/field/R112125}{Machine Learning}}

\begin{document}

\maketitle

\begin{abstract}
Twitter is one of the most popular social networks. \researchproblem{Previous research found that users employ Twitter to communicate about software applications via short messages, commonly referred to as tweets, and that these tweets can be useful for
requirements engineering and software evolution. However, due to their large number—in the range of thousands per day for
popular applications—a manual analysis is unfeasible}. In this work we present \objective{ALERTme, an approach to automatically classify, group and rank tweets about software applications}. We apply \method{machine learning techniques for automatically classifying tweets requesting improvements, topic modeling for grouping semantically related tweets and a weighted function for ranking tweets according to specific attributes, such as content category, sentiment and number of retweets}. We ran our approach on 68,108 collected tweets from three software applications and compared its results against software practitioners’ judgement. Our results show that \conclusion{ALERTme is an effective approach for filtering, summarizing and ranking tweets about software applications}. ALERTme enables the exploitation of Twitter as a feedback channel for information relevant to software evolution, including end-user requirements.
\end{abstract}

\end{document}
