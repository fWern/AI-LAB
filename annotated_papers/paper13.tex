\documentclass{article}

\usepackage[colorlinks=true, allcolors=blue]{hyperref}
\usepackage[compatibility]{scikgtex}

\title{
% add title to metadata
  \metatitle{SAFE: A Simple Approach for Feature Extraction
from App Descriptions and App Reviews}
}
\author{
% add author to metadata
  \metaauthor{Timo Johann}
  \and
  \metaauthor{Christoph Stanik}
  \and
  \metaauthor{Alireza M.}
  \and
  \metaauthor{Alizadeh B.}
  \and
  \metaauthor{Walid Maalej}
}

\begin{document}

\maketitle

\begin{abstract}
A main advantage of app stores is that they aggregate important information created by both developers and users. In the app store product pages, developers usually describe and maintain the features of their apps. In the app reviews, users comment these features. Recent studies focused on mining app features either as described by developers or as reviewed by users. However, \researchproblem{extracting and matching the features from the app descriptions and the reviews is essential to bear the app store advantages, e.g. allowing analysts to identify which app features are actually being reviewed and which are not}. In this paper, we \objective{propose SAFE, a novel uniform approach to extract app features from the single app pages, the single reviews and to match them}. We \method{manually build 18 part-of-speech patterns and 5 sentence patterns that are frequently used in text referring to app features. We then apply these patterns with several text pre- and post-processing steps}. A major advantage of our approach is that it does not require large training and configuration data. To evaluate its accuracy, we manually extracted the features mentioned in the pages and reviews of 10 apps. The \conclusion{extraction precision and recall outperformed two state-of-the-art approaches}. For well-maintained app pages such as for Google Drive our approach has \result{a precision of 87\% and on average 56\% for 10 evaluated apps. SAFE also matches 87\% of the features extracted from user reviews to those extracted from the app descriptions}.
\end{abstract}

\end{document}
