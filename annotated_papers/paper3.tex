\documentclass{article}

\usepackage[colorlinks=true, allcolors=blue]{hyperref}
\usepackage[compatibility]{scikgtex}

\title{
% add title to metadata
  \metatitle{Automated Question Answering for
Improved Understanding of Compliance
Requirements: A Multi-Document Study}
}
\author{
% add author to metadata
  \metaauthor{Sallam Abualhaija}
  \and
  \metaauthor{Chetan Arora}
  \and
  \metaauthor{Amin Sleimi}
  \and
  \metaauthor{Lionel C. Briand}
}
\researchfield*{\uri{https://orkg.org/field/R145261}{Natural Language Processing}}

\begin{document}

\maketitle

\begin{abstract}
Software systems are increasingly subject to regulatory compliance. \researchproblem{Extracting compliance requirements from regulations is challenging. Ideally, locating compliance-related information in a regulation requires a joint effort from requirements engineers and legal experts, whose availability is limited}. However, regulations are typically long documents spanning hundreds of pages, containing legal jargon, applying complicated natural language structures, and including cross-references, thus making their analysis effort-intensive. In this paper, we propose \objective{an automated question-answering (QA) approach that assists requirements engineers in finding the legal text passages relevant to compliance requirements. Our approach utilizes large-scale language models finetuned for QA, including BERT and three variants}. We evaluate our \method{approach on 107 question-answer pairs, manually curated by subject-matter experts, for four different European regulatory documents}. Among these documents is the general data protection regulation (GDPR) – a major source for privacy-related requirements. Our empirical \result{results show that, in ≈94\% of the cases, our approach finds the text passage containing the answer to a given question among the top five passages that our approach marks as most relevant}. Further, \result{our approach successfully demarcates, in the selected passage, the right answer with an average accuracy of ≈91\%}.
\end{abstract}

\end{document}
