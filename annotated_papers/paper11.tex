\documentclass{article}

\usepackage[colorlinks=true, allcolors=blue]{hyperref}
\usepackage[compatibility]{scikgtex}

\title{
% add title to metadata
  \metatitle{Mining Binary Constraints in the Construction of Feature Models}
}
\author{
% add author to metadata
  \metaauthor{Li Yi}
  \and
  \metaauthor{Wei Zhang}
  \and
  \metaauthor{Haiyan Zhao}
  \and
  \metaauthor{Zhi Jin}
  \and
  \metaauthor{Hong Mei}
}

\begin{document}

\maketitle

\begin{abstract}
Feature models provide an effective way to organize and reuse requirements in a specific domain. A feature model consists of a feature tree and cross-tree constraints. Identifying features and then building a feature tree takes a lot of effort, and many semi-automated approaches have been proposed to help the situation. However, \researchproblem{finding cross-tree constraints is often more challenging which still lacks the help of automation}. In this paper, we propose an \objective{approach to mining cross-tree binary constraints in the construction of feature models}. Binary constraints are the most basic kind of cross-tree constraints that involve exactly two features and can be further classified into two sub-types, i.e. requires and excludes. Given these two sub-types, a pair of any two features in a feature model falls into one of the following classes: no constraints between them, a requires between them, or an excludes between them. Therefore we perform a \method{3-class classification on feature pairs to mine binary constraints from features. We incorporate a \uri{https://orkg.org/resource/R3096}{support vector machine} as the classifier and utilize a \uri{https://orkg.org/resource/R3072}{genetic algorithm} to optimize it. We conduct a series of experiments on two feature models constructed by third parties, to evaluate the \uri{https://orkg.org/resource/R68581}{effectiveness} of our approach under different conditions that might occur in practical use}. Results show that we can mine binary constraints at a \result{high recall (near 100\% in most cases)}, which is important because finding a missing constraint is very costly in real, often large, feature models.
\end{abstract}

\end{document}
